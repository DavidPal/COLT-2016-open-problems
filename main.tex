\documentclass{colt2016} % Include author names

\usepackage[nolist]{acronym}
\usepackage{algorithm,algorithmic}
\usepackage{times}
\usepackage{enumerate}

\DeclareMathOperator{\Regret}{Regret}
\DeclareMathOperator{\polylog}{polylog}

\newcommand{\R}{\mathbb{R}}     % real numbers
\renewcommand{\H}{\mathcal{H}}  % Hilbert space
\newcommand{\KL}[2]{D\left({#1}\middle\|{#2}\right)}  % KL divergence
\newcommand{\norm}[1]{\left\|{#1}\right\|}
\DeclareMathOperator*{\argmin}{arg\,min}
\newcommand{\indicator}{\mathbf{1}}

\coltauthor{%
   \Name{Francesco Orabona} \Email{francesco@orabona.com}\\
   \Name{D\'avid P\'al} \Email{dpal@yahoo-inc.com}\\
{\addr Yahoo Research, New York}
}

\title{Open Problem: Parameter-Free and Scale-Free Online Algorithms}

\begin{document}

\maketitle

\begin{abstract}
Existing vanilla algorithms for online linear optimization have $O((\eta R(u) +
1/\eta) \sqrt{T})$ regret with respect to any competitor $u$, where $R(u)$ is a
$1$-strongly convex regularizer and $\eta > 0$ is a tuning parameter of the
algorithm. For certain decision sets and regularizers, the so-called
\emph{parameter-free} algorithms have $\widetilde O(\sqrt{R(u) T})$ regret with
respect to any competitor $u$.  Vanilla algorithm can achieve the same bound
only for a fixed competitor $u$ known ahead of time by setting $\eta =
1/\sqrt{R(u)}$. A drawback of both vanilla and parameter-free algorithms is
that they assume that the norm of the loss vectors is bounded by a constant
known to the algorithm. There exist \emph{scale-free} algorithms that have
$O((\eta R(u) + 1/\eta) \sqrt{T} \max_{1 \le t \le T} \norm{\ell_t})$ regret
with respect to any competitor $u$ and for any sequence of loss vector $\ell_1,
\dots, \ell_T$. Parameter-free analogue of scale-free algorithms have never
been designed. Is is possible to design algorithms that are simultaneously
\emph{parameter-free} and \emph{scale-free}?
\end{abstract}

\section{Introduction}

Online linear optimization (OLO)~\citep{Cesa-Bianchi-Lugosi-2006,
Shalev-Shwartz-2011} is a sequential decision making problem where, in each
round $t$, an algorithm chooses a point $x_t$ from a convex \emph{decision set}
$K$ and then receives a loss vector $\ell_t$. Algorithm's goal is to keep its
cumulative loss $\sum_{t=1}^T \langle \ell_t, x_t \rangle$ small. Algorithm can
be evaluated by comparing its loss with the loss of hypothetical strategy that
in every round chooses the same point $u$; the difference of two losses is
called \emph{regret}. More formally, \emph{regret with respect to a competitor
$u \in K$ after $T$ rounds} is
$$
\Regret_T(u) = \sum_{t=1}^T \langle \ell_t, x_t \rangle - \sum_{t=1}^T \langle \ell_t, u \rangle \; .
$$

Algorithms for various decision sets have been investigated, e.g., the
probability simplex, various combinatorial polytopes, Hilbert space, unit ball
in Hilbert space. We focus on two particular sets,\footnote{We leave the
generalization of the open problem to arbitrary decision sets as an exercise
for the reader.} the $N$-dimensional probability simplex $\Delta_N = \{ x \in
\R^N ~:~ x \ge 0, \norm{x}_1 = 1\}$ and the Hilbert space. OLO over $\Delta_N$
is referred to as the problem of Learning with Expert Advice (LEA); it used as
a way of combining $N$ predictors and in boosting. OLO over Hilbert space is
the workhorse for learning of generalized linear models in very high dimension.

We will use the following notation in the rest of this note. We denote by
$\indicator$ the vector $(1,1,\dots,1) \in \R^N$. Shannon entropy $H(u) =
-\sum_{i=1}^N u_i \ln u_i$ is defined for any $u \in \Delta_N$.  The
Kullback-Leibler divergence $\KL{u}{v} = \sum_{i=1}^N u_i \ln(u_i/v_i)$ is
defined for any $u,v \in \Delta_N$. For any $p \in [1,\infty]$,
$\norm{\cdot}_p$ denotes $p$-norm in $\R^N$.  We denote by $\norm{\cdot}_*$ the
dual norm of a norm $\norm{\cdot}$.  If $\H$ is a real Hilbert space, we denote
by $\langle \cdot, \cdot \rangle$ its inner product, and by $\norm{\cdot}$ the
induced norm.

We will use the following well know facts.  Negative Shannon entropy, $-H(u)$,
defined on $\Delta_N$ is $1$-strongly convex with respect to $\norm{\cdot}_1$.
The dual norm of $\norm{\cdot}_1$ is $\norm{\cdot}_\infty$.  The function $R(u)
= \frac{1}{2}\norm{u}^2$ defined on Hilbert space with norm $\norm{\cdot}$ is
$1$-strongly convex with respect to $\norm{\cdot}$.

Follow The Regularized Leader (FTRL) algorithm with regularizer $R:K \to \R$
and \emph{learning rate} $\eta > 0$ in round $t$ chooses
$
x_t = \argmin_{x \in K} \left( \frac{1}{\eta} R(x) + \sum_{s=1}^{t-1} \langle \ell_s, x \rangle \right)
$.
The following theorem is a slight modification of \citet[Theorem 2.11]{Shalev-Shwartz-2011}.
\begin{theorem}[Regret of FTRL]
\label{theorem:ftrl-regret}
If $R:K \to \R$ is $1$-strongly convex function with respect a norm
$\norm{\cdot}$ then for any sequence $\{\ell_t\}_{t=1}^\infty$ such that
$\|\ell_t\|_* \le 1$, FTRL with learning rate $\eta$ satisfies
$$
\forall T \ge 0 \quad \forall u \in K \qquad \Regret_T(u) \le \frac{R(u) - \inf_{v \in K} R(v)}{\eta} + \frac{\eta T}{2} \; .
$$
\end{theorem}

\section{Learning with Expert Advice}

Hedge algorithm~\citep{Freund-Schapire-1997} for LEA satisfies
\begin{equation}
\label{equation:hedge-bound}
\forall u \in \Delta_N \qquad \Regret_T(u) \le \sqrt{2 T \ln N}
\end{equation}
This bound is known to be optimal in the worst-case sense~\cite[Section
3.7]{Cesa-Bianchi-Lugosi-2006}. However, \eqref{equation:hedge-bound} has two
drawbacks.  First, the right-hand side of \eqref{equation:hedge-bound} is
independent of $u$, that is, the algorithm does not adapt to $u$.
Second, Hedge satisfies \eqref{equation:hedge-bound} only if $\ell_1, \ell_2,
\dots, \ell_T \in [-1,1]^N$. We would like to allow loss vectors that are
arbitrary vectors in $\R^N$.

Hedge is identical to FTRL with regularizer $R(u) = -H(u)$.
Theorem~\ref{theorem:ftrl-regret} implies that Hedge with learning rate $\eta$, satisfies
\begin{equation}
\label{equation:hedge-bound-2}
\Regret_T(u) \le \frac{\ln N - H(u)}{\eta} + \frac{\eta T}{2} = \frac{\KL{u}{\frac{1}{N}\indicator}}{\eta} + \frac{\eta T}{2} \; .
\end{equation}
Let $p \in [0, \ln N)$. If we choose $\eta = \sqrt{\frac{\ln(N) - p}{T}}$, we get that
\begin{equation}
\label{equation:hedge-bound-3}
\forall u \in \Delta_N, \qquad H(u) \ge p \quad \Longrightarrow \quad \Regret_T(u) \le \sqrt{2 T (\ln(N) - p)} \; .
\end{equation}
The bound \eqref{equation:hedge-bound} corresponds to choice $p=0$ and
$\eta=\sqrt{\frac{\ln N}{T}}$.

Instead of the family of algorithms parametrized by $p \in [0,\ln N)$ that
satisfy bound~\eqref{equation:hedge-bound-3}, one \emph{would like to have} a single
algorithm (without any tuning parameters) satisfying
\begin{equation}
\label{equation:parameter-free-bound-experts}
\forall u \in \Delta_N \qquad \Regret_T(u) \le \sqrt{2 T (\ln N - H(u))} = \sqrt{2T \cdot \KL{u}{\tfrac{1}{N} \indicator}} \; .
\end{equation}
Note that \eqref{equation:parameter-free-bound-experts} is stronger than
\eqref{equation:hedge-bound-3} in following the sense: A single algorithm
satisfying~\eqref{equation:parameter-free-bound-experts} implies
\eqref{equation:hedge-bound-3} for all $p$ simultaneously. However, a family
of algorithms $\{A_p ~:~ p \in [0,\ln N)\}$ parametrized by $p$ where $A_p$
satisfies \eqref{equation:hedge-bound-3}, does not yield a single
algorithm satisfying \eqref{equation:parameter-free-bound-experts}.

Bounds of the form~\eqref{equation:parameter-free-bound-experts} were not
considered until 2009. Since then, however, there have been a lot of work
\citep{Chaudhuri-Freund-Hsu-2009, Chernov-Vovk-2010, Koolen-van-Erven-2015,
LuoS14,Luo-Schapire-2015, Foster-Rakhlin-Sridharan-2015,
Orabona-Pal-2016-parameter-free} on algorithms that satisfy slightly
looser\footnote{Earlier papers have extra logarithmic factors.
\citet{Foster-Rakhlin-Sridharan-2015, Orabona-Pal-2016-parameter-free} have
only the extra $1+$ and fixed multiplicative constant hidden on $\widetilde
O(\cdot)$} versions of~\eqref{equation:parameter-free-bound-experts}
\begin{equation}
\label{equation:parameter-free-bound-experts-2}
\forall u \in \Delta_N \qquad \Regret_T(u) \le \widetilde O(\sqrt{T (1 + \ln N - H(u))}) = \widetilde O\left(\sqrt{T(1 + \KL{u}{\tfrac{1}{N}\indicator} )} \right) \; .
\end{equation}
Algorithms of this type are called \emph{parameter-free} since, in contrast to
Hedge, they do not need to know $p$.  The new algorithms are also called
\emph{algorithms for unknown number of experts},
and~\eqref{equation:parameter-free-bound-experts} is called a \emph{quantile
bound}. The reason for these names is that in order to bound the regret with
respect to $(\epsilon N)$-th best expert for some $\epsilon \in (0,1)$, we
bound the regret with respect to the average of the best $\epsilon N$ experts.
That is, up to permutation of coordinates, we consider competitors $u$ of the
form
$$
u = \left( 1/(\epsilon N), \dots, 1/(\epsilon N), 0, \dots, 0 \right) \; .
$$
Such competitor satisfies $H(u) = \ln (\epsilon N)$ and the regret with respect
to any such $u$ is $\widetilde O(\sqrt{T (1 + \ln(1/\epsilon)})$. In
particular, the last bound does not depend on the number of experts $N$, only
on the quantile $\epsilon$.  These algorithms remove the first drawback of
Hedge.

The second drawback of Hedge is removed by the AdaHedge algorithm due to
\cite{de-Rooij-van-Erven-Grunwald-Koolen-2014}; see
also~\citep{Orabona-Pal-2016-parameter-free}. AdaHedge lifts the assumption
$\ell_t \in [-1,1]^N$. Namely, for any sequence of loss vectors
$\{\ell_t\}_{t=1}^\infty$, $\ell_t \in \R^N$, any $T \ge 0$ and any $u \in
\Delta_N$, AdaHedge satisfies $\Regret_T(u) \le 5.3 \sqrt{\ln N \sum_{t=1}^T
\norm{\ell_t}_\infty^2}$. AdaHedge is \emph{scale-free} which means that its
predictions $x_t$ are the same for $\{\ell_t\}_{t=1}^\infty$ and $\{c
\ell_t\}_{t=1}^\infty$ where $c$ is any positive constant.

Our first open problem is to design an algorithm that combines the advantages
of parameter-free and scale-free algorithms. More formally: \emph{Does there
exist a universal constant $C > 0$ and, for each $N \ge 2$, is there an
algorithm (without any tuning parameters) such that for any sequence of loss
vectors $\{\ell_t\}_{t=1}^\infty$, $\ell_t \in \R^N$,}
$$
\forall T \ge 0 \quad \forall u \in \Delta_N \qquad \Regret_T(u) \le C \sqrt{(1 + \ln N - H(u)) \sum_{t=1}^T \norm{\ell_t}_\infty^2} \ \ ?
$$
It is probably a good idea to restrict the search to scale-free algorithms,
since non-scale-free algorithms are unlikely to satisfy the bound.

\section{OLO over Hilbert Spaces}

The situation with algorithms for OLO over a Hilbert space $\H$ is very similar
to that of LEA.

FTRL with regularizer
$\frac{1}{2}\norm{u}^2$ and learning rate $\eta$ satisfies (cf. Theorem~\ref{theorem:ftrl-regret})
\begin{equation}
\label{equation:ftrl-vanila}
\forall u \in \H \qquad \Regret_T(u) \le \frac{\norm{u}^2}{2\eta} + \frac{\eta T}{2},
\end{equation}
assuming that $\norm{\ell_1}, \norm{\ell_2}, \dots, \norm{\ell_T} \le 1$.
Bound~\eqref{equation:ftrl-vanila} is a direct analogue of
\eqref{equation:hedge-bound-2} for LEA.

A simple choice $\eta = 1/\sqrt{T}$ leads to an algorithm that satisfies
\begin{equation}
\label{equation:ftrl-vanila-2}
\Regret_T(u) \le \frac{1}{2}\left(1+\norm{u}^2\right)\sqrt{T} \; .
\end{equation}
However, this algorithm and the bound \eqref{equation:ftrl-vanila-2} have two
drawbacks.  First, the dependency on $\norm{u}$ is suboptimal. As we will see
shortly, the quadratic dependency can be replaced by an (almost) linear
dependency.  Second, the bound holds only for sequences of loss vectors with
$\norm{\ell_t} \le 1$, $t=1,2,\dots,T$. A robust algorithm should be able to
handle any sequence of loss vectors in $\H$.

Starting from \eqref{equation:ftrl-vanila}, if we choose learning rate $\eta =
D/\sqrt{T}$, we get a family of algorithms parametrized by $D \in [0,\infty)$
that satisfy an analogue bound of \eqref{equation:hedge-bound-3}:
\begin{equation}
\label{equation:ftrl-vanila-3}
\forall u \in \H \qquad \norm{u} \le D \quad  \Longrightarrow \quad \Regret_T(u) \le D \sqrt{T} \; .
\end{equation}

Instead of family of algorithms parametrized by $D \in [0,\infty)$ satisfying
bound \eqref{equation:ftrl-vanila-3}, one \emph{would like
to have} a single algorithm (without any tuning parameters) satisfying
\begin{equation}
\label{equation:olo-parameter-free}
\forall u \in \H \qquad \Regret_T(u) \le \norm{u} \sqrt{T} \; .
\end{equation}
Bound \eqref{equation:olo-parameter-free} is an analogue of
\eqref{equation:parameter-free-bound-experts} for LEA. Similar to LEA,
\eqref{equation:olo-parameter-free} is stronger than
\eqref{equation:ftrl-vanila-3} in the following sense: A single algorithm
satisfying \eqref{equation:olo-parameter-free} implies
\eqref{equation:ftrl-vanila-3} for all values of $D \in [0,\infty)$.  However,
a family of algorithms $\{A_D : D \in [0,\infty)\}$ parametrized by $D$ where
$A_D$ satisfies \eqref{equation:ftrl-vanila-3}, does not yield a single
algorithm that satisfies \eqref{equation:olo-parameter-free}.  Finally, note
that \eqref{equation:olo-parameter-free} has better dependency on $\norm{u}$
than \eqref{equation:ftrl-vanila-2}.

Similar to LEA, there have been a lot of work on algorithms
\citep{Streeter-McMahan-2012, Orabona-2013, McMahan-Abernethy-2013,
McMahan-Orabona-2014} that satisfy a slightly weaker version of
\eqref{equation:olo-parameter-free}
\begin{equation}
\label{equation:olo-parameter-free-2}
\forall u \in \H \qquad \Regret_T(u) \le \sqrt{\norm{u} T} \polylog(1 + \norm{u}, T) \; ,
\end{equation}
where $\polylog(1 + \norm{u}, T)$ represents a function that is upper bounded
by a polynomial in $\log(1+\norm{u})$ and $\log T$.\footnote{It can be shown
that for OLO over Hilbert space extra poly-logarithmic factor is
necessary~\citep{McMahan-Abernethy-2013,Orabona-2013}.} Algorithms satisfying
\eqref{equation:olo-parameter-free-2} are called \emph{parameter-free}, since
they do not need to know $D$.  Moreover, the bound
\eqref{equation:olo-parameter-free-2} has much better dependency on $\norm{u}$
than \eqref{equation:ftrl-vanila-2}.

Analogue of AdaHedge for OLO over Hilbert space is FTRL with adaptive learning
rate $\eta_t = 1/\sqrt{\sum_{i=1}^{t-1} \norm{\ell_i}^2}$.
\cite{Orabona-Pal-2015} (see also \cite{Orabona-Pal-2016-scale-free}) showed
that the resulting algorithm is scale-free and for any sequence of loss vectors
$\{\ell_t\}_{t=1}^\infty$, $\ell_t \in \H$, it satisfies
$$
\forall T \ge 0 \quad \forall u \in \H \qquad \Regret_T(u) \le \left(6.25 + \frac{1}{2}\norm{u}^2 \right) \sqrt{T} \max_{1 \le t \le T} \norm{\ell_t} \; .
$$
We stress that the algorithm does \emph{not} need to know $\max_{1 \le t \le T}
\norm{\ell_t}$.

Our second open problem is to design an algorithm that combines the advantages
of parameter-free and scale-free algorithms for OLO over a Hilbert space $\H$.
Formally: \emph{Is there an algorithm (without any tuning parameters) such that
for any sequence of loss vectors $\{\ell_t\}_{t=1}^\infty$, $\ell_t \in \H$,}
$$
\forall T \ge 0 \quad \forall u \in \H \qquad
\Regret_T(u) \le \polylog(1 + \norm{u}, T) \cdot (1 + \norm{u}) \sqrt{T} \max_{1 \le t \le T} \norm{\ell_t} \ \ ?
$$
Here $\polylog(1 + \norm{u}, T)$ represents a function that is upper bounded by
a polynomial in $\log(1+\norm{u})$ and $\log T$. It is probably a good idea to
restrict the search to scale-free algorithms, since non-scale-free algorithms
are unlikely to satisfy the bound. The problem is interesting even for
one-dimensional Hilbert space $\H = \R$. Also, the reductions in
\cite{Orabona-Pal-2016-parameter-free} suggest that solving the
one-dimensional problem solves the general Hilbert space and the LEA cases.

\bibliography{biblio}

\end{document}
